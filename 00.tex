\chapter*{Introduction}
\begin{center}
Le note sono in inglese per coerenza con la lingua in cui il corso \`e stato tenuto.    
\end{center}

\section*{What is the course about?}
The first part of the course deals with:
\begin{itemize}
    \item Algebraic Tori, their actions and representations
    \item Affine toric varieties (with monoids) $\leftrightarrow$ cones in some $\R^n$
    \item Projective toric varieties $\leftrightarrow$ polytopes in some $\R^n$
    \item General toric varieties $\leftrightarrow$ fans in $\R^n$
\end{itemize}
We will then deal with (suject to change)
\begin{itemize}
    \item Divisors/line bundles on toric varieties
    \item Cox ring of a toric variety
    \item Cohomology of divisors
    \item Toric morphisms and resolution of singularities
    \item and more!
\end{itemize}


Let us set some ground assumptions: we'll work over an algebraically closed field (and we will be lax about the characteristic of the field). The main reference for the course (Cox, Little, Schenk ``Toric varieties") works over $\C$ but a lot of things work more generally.


\begin{definition}[Toric variety]
An $n$-dimensional toric variety $X$ is a (normal) $k$-variety equipped with an open immersion of an $n$-dimensional torus $T\subseteq X$, where $T\cong (k^\ast)^n$, and an action $T\times T\to T$ which extends to the whole of $X$ \footnote{that is, it extends to a $T\times X\to X$}.
\end{definition}

\begin{remark}
Normality is a standard assumption that we'll make at some point but some things work without it.
\end{remark}

We'll see that the geometry of such an object is encoded in a combinatorial gadgets, reducing problems in algebraic geometry to problems in combinatorics, which is sometimes convenient.

Sometimes the opposite happens and results in combinatorics can be proven using algebraic geometry
\begin{example}[McMullen's ``$g$-conjecture"]
The theorem is a characterization of the $f$-vectors of simple polytopes\footnote{for now, convex hull of a finite subset of $\R^n$}.

\begin{definition}[$f$-vectors]
If $P$ is a polytope, its \textbf{$f$-vector} is 
\[(f_0(P),\cdots, f_d(P)),\quad \text{where $d=\dim P$}\]
and $f_i(P)$ is the number of $i$-dimensional faces. We may set $f_{-1}(P)=1$.
\end{definition}

It's reasonable to ask ourselves which $f$-vectors can appear. We may define the $h$-vector by setting
\[\sum_{i=0}^d f_i (t-1)^i=\sum_{i=0}^dh_i t^i,\quad \text{i.e. }h_i=\sum_{j=i}^d(-1)^{j-i}\binom{j}if_j,\ h_{-1}=0\]
It is a theorem that the $h$-vector of a simple polytope is palindromic ($h_i=h_{d-i}$).

We obtain the \textbf{$g$-vector} by setting $g_i=h_i-h_{i-1}$. The conjecture was that
\begin{theorem}[$g$-conjecture]
$f=(f_0,\cdots, f_d)\in \N^{d+1}$ is the $f$-vector of a simple polytope if
\begin{enumerate}
\item $h_i=h_{d-1}$ for all $0\leq i\leq \floor{d/2}$
\item $g_i\geq 0$ for all $0\leq i\leq \floor{d/2}$
\item $(g_1,\cdots, g_{\floor{d/2}})$ is a ``Macauly vector" if, when we write (uniquely)
\[g_i=\binom{n_i}i+\cdots+\binom{n_{r_i}}{r_i}\]
with $n_i>n_{i-1}>\cdots>n_{r_i}$ then
\[g_{i+1}\leq \binom{n_i+1}{i+1}+\cdots+\binom{n_{r_i}+1}{r_i+1}\]
\end{enumerate}
\end{theorem}
Stanley proved necessity using toric varieties. He proved that the $g$-vector of a simple polytope is the vector of dimensions for some cohomology ring of the associated toric variety.


Later McMullen found a completely combinatorial proof.
\end{example}