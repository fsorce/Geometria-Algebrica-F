\chapter{Projective toric varieties}

\section{Introduction}
\begin{definition}[]
A \textbf{projective toric variety} is an irreducible, normal projective variety $X$ equipped with an open embedding $T\subseteq X$ of an algebraic torus such that the translation action of $T$ extends to $X$.
\end{definition}

\begin{remark}
Projective space $\Pj^n$ is a projective toric variety with torus given by
\[\Pj^n\bs V(x_0\cdots x_n).\]
This is the same torus that we get on all the affine charts.

The translation action extends as follows:
\[\funcDef{\G_m^n\times \Pj^n}{\Pj^n}{((a_1,\cdots, a_n),[x_0,\cdots, x_n])}{[x_0,a_1x_1,\cdots, a_n x_n]}\]
The character lattice of this torus $T_{\Pj^n}$ can be thought of as follows: recall that we have
\[\A^{n+1}\nz\xrightarrow{\pi}\frac{\A^{n+1}\nz}{\G_m}\cong \Pj^n\]
and this induces
\[0\to \G_m\to \G_m^{n+1}\to T_{\Pj^n}\to0\]
where the first inclusion is via matricies of the form $\la I$. Dually we get a short exact sequence of the character lattices
\[0\to M_{\Pj^n}\to \Z^{n+1}\to \Z\to 0\]
so we may write
\[M_{\Pj^n}=\cpa{(a_0,\cdots, a_n)\in \Z^{n+1}\mid \sum a_i=0}\subseteq \Z^{n+1}.\]
\end{remark}

Now, given a finte subset $A\subseteq M$ (let us write $A=\cpa{a_1,\cdots, a_s}$) we can consider
\[\vp_A:\funcDef{T_N}{\G_m^s}{t}{(\chi^{a_1}(t),\cdots, \chi^{a_s}(t))}\]
and then the composition
\[\psi_A:T_N\xrightarrow{\vp_A}\G_m^s\inj \A^2\nz\onto \Pj^{s-1}.\]
The closure of the image of $\psi_A$ inside $\Pj^{s-1}$ is the \textbf{projective toric variety $X_A$ associated to $A$}

\begin{proposition}[]\label{PrProjectiveToricVarietyAssociatedToFiniteSubsetOfCharacters}
$X_A$ as above is a projective toric variety. $\dim X_A=\dim \mathrm{Affspan}_\R{A}$ where the last notation means \textit{the affine subspace generated by $A$} in $M_\R$.
\end{proposition}
\begin{proof}
Let $T$ be the image of $T_N\to \G_m^s\to T_{\Pj^{s-1}}$, which is still a torus by (\ref{PrImageOfTorusInATorusIsATorus}). Note that $X_A$ is the closure of $T$ in $\Pj^{s-1}$.

If $t\in T$, $t\cdot T=T\subseteq X_A$ and $\ol{t\cdot T}=t\cdot \ol T=t\cdot X_A$, so $t\cdot X_A\subseteq X_A$, but the same holds for $t\ii$, thus the action extends.

$\dim X_A=\dim T=\rnk_\Z M'$ where $M'=X(T)$. We can compute $M'$:
% https://q.uiver.app/#q=WzAsMyxbMCwwLCJUX04iXSxbMSwwLCJUIl0sWzEsMSwiVF97XFxQal57cy0xfX0iXSxbMCwyXSxbMSwyLCIiLDMseyJzdHlsZSI6eyJ0YWlsIjp7Im5hbWUiOiJob29rIiwic2lkZSI6InRvcCJ9fX1dLFswLDEsIiIsMyx7InN0eWxlIjp7ImhlYWQiOnsibmFtZSI6ImVwaSJ9fX1dXQ==
\[\begin{tikzcd}
	{T_N} & T \\
	& {T_{\Pj^{s-1}}}
	\arrow[two heads, from=1-1, to=1-2]
	\arrow[from=1-1, to=2-2]
	\arrow[hook, from=1-2, to=2-2]
\end{tikzcd}\]
yields dually (apply $X(\cdot)$ functor)
% https://q.uiver.app/#q=WzAsMyxbMSwxLCJNX3tcXFBqXntzLTF9fSJdLFsxLDAsIk0nIl0sWzAsMCwiTSJdLFswLDJdLFsxLDIsIiIsMyx7InN0eWxlIjp7InRhaWwiOnsibmFtZSI6Imhvb2siLCJzaWRlIjoiYm90dG9tIn19fV0sWzAsMSwiIiwzLHsic3R5bGUiOnsiaGVhZCI6eyJuYW1lIjoiZXBpIn19fV1d
\[\begin{tikzcd}
	M & {M'} \\
	& {M_{\Pj^{s-1}}}
	\arrow[hook', from=1-2, to=1-1]
	\arrow[from=2-2, to=1-1]
	\arrow[two heads, from=2-2, to=1-2]
\end{tikzcd}\]
so $M'$ is the image of $M_{\Pj^{s-1}}\to M$, which is induced by the map $\Z^s\to M$ which sends $e_i$ to $a_i$, so the image is exactly
\[\cpa{\sum k_i a_i\mid \sum k_i=0}=\ps{a_i-a_j\mid i\neq j}\subseteq M.\]
Upon tensoring this with $\R$ we get the vector subspace of $M_\R$ associated to the affine subspace generated by $A$.
\end{proof}


\begin{remark}
One may expect $Y_A\subseteq \A^{s}$ to be related to the affine cone over $X_A$. The two are the same if and only if $I(Y_A)$ is homogeneous iff exists $n\in N$ and $k$ positive such that **************
(i.e. $A$ is contained in an affine hyperplane of $M_\R$).
\end{remark}

\begin{remark}
The toric variety $X_A\subseteq \Pj^{s-1}$ is covered by affine toric varieties, given by the intersections $X_A\cap U_i$. The $X_A\cap U_i$ are indeed affine and they are toric because they all contain $T$. In fact $X_A\cap U_i=\ol{T}^{U_i}$.
\end{remark}


\begin{proposition}[]\label{PrMonoidOfAffineChartOfProjectiveToricVariety}
The monoid of $X_A\cap U_i$ is the submonoid $A_i$ of $M$ generated by $a_j-a_i$ for $j\neq i$.
\end{proposition}
\begin{proof}
It suffices to show that $X_A\cap U_i$ is the closure of the image of $T_N\to U_i\to \A^{s-1}$. If $t\in T_N$ then the maps go
\[t\mapsto [\chi^{a_1}(t),\cdots,\chi^{a_s}(t)]\mapsto (\chi^{a_1-a_i}(t),\cdots, \chi^{a_s-a_i}(t))\]
and this is exactly what we want.
\end{proof}


\begin{remark}
$A_i^{gp}$ is exactly the character lattice of $T$ that we found the proof before.
\end{remark}



\begin{example}[Rational normal curve]
Let $A\subseteq \Z^2$ be the subset given by $A=\cpa{(0,d),(1,d-1),\cdots, (d,0)}$. The affine toric variety $Y_A$ is what we called \textit{rational normal cone of degree $d$}.

The projective toric variety $X_A$ is called the \textbf{rational normal curve of degree $d$} in $\Pj^d$ and $Y_A$ is its affine cone in $\A^{d+1}$.
\end{example}

\begin{example}
Let $A=\cpa{e_1,e_2,e_3,e_1+e_2-e_3}$. The affine toric variety is
\[Y_A=\Spec\frac{k[x,y,z,w]}{(xy-zw)}\subseteq\A^4\]
The projective toric variety $X_A$ is the one in $\Pj^3$ given by the same equation $xy=zw$. This is actually isomorphic to $\Pj^1\times \Pj^1$ via the Segre embedding
\[\funcDef{\Pj^1\times \Pj^1}{\Pj^3}{([y_0,y_1],[z_0,z_1])}{[y_0z_0,y_1z_1,y_0z_1,y_1z_0]}\]
\end{example}


\section{Polytopes}
We have seen that affine toric varieties are described by cones. For projective toric varieties we have a similar correspondence with polytopes.

\begin{definition}[]
A \textbf{polytope} in $M_\R$ is the convex hull of a finite subset $A\subseteq M_\R$, i.e.
\[P=\Conv(A)=\cpa{\sum_{a\in A}\la_a a\mid \la_a\geq 0,\ \sum \la_a=1}.\]
\end{definition}

Given such a $P$ we can construct a cone
\[\Cone(A\times \cpa{1})\subseteq M_\R\oplus \R.\]
We can recover the polytope by slicing the cone at height 1.

This corresponcence is sometimes useful to prove things about polytopes by reducing to the case of cones.

\begin{definition}[]
The \textbf{dimension} of a polytope $P$ is the dimension of the smalled affine subspace of $M_\R$ which contains $P$.
\end{definition}

\begin{definition}[]
Let $u\in N_\R$ and $b\in \R$. They determine an \textbf{affine hyperplane}
\[H_{u,b}=\cpa{m\in M_\R\mid \ps{m,u}=b}\subseteq M_\R\]
and a \textbf{closed half-space}
\[H_{u,b}^+=\cpa{m\in M_\R\mid \ps{m,u}\geq b}\subseteq M_\R\]
\end{definition}

\begin{definition}[]
A subset $Q\subseteq P$ is a \textbf{face} if there exist $n\in N_\R$ and $b\in \R$ such that $P\subseteq H_{u,b}^+$ (in this case we say that $H_{u,b}$ is a \textbf{supporting hyperplane}) and $Q=P\cap H_{u,b}$.
\end{definition}

\begin{remark}
Faces of a polytope are polytopes. Moreover, if $P=\Conv(A)$ then $Q=\Conv(A\cap H_{u,b})$ for $H_{u,b}$ supporting hyperplane which defines $Q$.
\end{remark}

\begin{definition}[]
Faces of dimension $0$ are called \textbf{vertices}, those of dimension 1 are \textbf{edges} and those of codimension 1 are \textbf{facets}.
\end{definition}

\begin{fact}
If $P$ is a polytope then
\begin{itemize}
\item $P=\Conv(\text{vertices of }P)$
\item If $P=\Conv(A)$ and $v\in P$ is a vertex then $v\in A$
\item if $Q\leq P$ then 
\[\cpa{\text{faces of }Q}=\cpa{\text{faces of $P$ contained in $Q$}}\]
\item if $Q<P$ (proper face) then 
\[Q=\bigcap_{\smat{F\text{ facet of $P$}\\ Q\leq F}}F\]
\item a polytope is a finite intersection of closed half-spaces
\item any finite intersection of closed half-spaces which is bounded is a polytope
\end{itemize}
\end{fact}

\begin{fact}
When $P$ is full-dimensional, each facet $F$ has a \textit{unique} supporting hyperplane.
\end{fact}

\begin{notation}
If $F$ is a facet of $P$ full-dimensional we use $H_F^+$ to denote the associated supporting hyperplane and we denote by $u_F\in N_\R$, $a_F\in \R$ the pair such that
\[H_F^+=H_{u_F,-a_F}.\]
The sign of $a_F$ is that way just for convention, it will make some computations easier later on. Note that the pair $(u_F,a_F)$ is not unique but it become unique up to positive scaling.
\end{notation}

\begin{definition}[]
A polytope $P$ is a \textbf{lattice polytope} if there exists $A\subseteq M$ finite such that $P=\Conv(A)$.
\end{definition}

\begin{remark}
This is equivalent to saying that all vertices of $P$ lie in $M$.
\end{remark}

\begin{fact}
The following propositions hold
\begin{itemize}
\item Faces of lattice polytopes are lattice polytopes
\item in the description of $P$ as $P=\bigcup_{i=1}^sH_{u_i,s_i}^+$ we can assume that the $u_i$ are also points in the lattice $N$
\item If $P$ is a full-dimensional lattice polytope we have a presentation 
\[P=\bigcup_{F\text{ facet of $P$}}H_F^+\]
and we can assume that $u_F$ is the minimal ray generator of $\Cone(u_F)$.
\item The presentation above for a given $P$ is unique and the pairs $(u_F,a_F)$ chosen as above ($u_F$ minimal ray generator) are unique. 
\[P=\cpa{m\in M_\R\mid \ps{m,u_F}\geq -a_F\ \forall F\text{ facet of }P}.\]
\end{itemize}
\end{fact}

\begin{example}
The standard $n$-dimensional simplex $\Delta_n=\Conv(0,e_1,\cdots, e_n)$ is a polytope of dimension $n$. It has exactly $n+1$ vertices.
\end{example}


\section{Toric varieties from polytopes}
Now the idea is, given a lattice polytope $P$, which we assume to be full-dimensional, is to take $X_{P\cap M}$

\begin{remark}
If $M$ is a lattice and $P$ is a lattice polytope in $M_\R$, $P\cap M$ is a finite set.
\end{remark}

This works, but if we want the combinatorics of $P$ to reflect the geometry of $X_{P\cap M}$ correctly, we need $P$ to have ``enough" lattice points.

There are two notions that are related to this issue: \textit{normality} and \textit{very ampleness}. We will only discuss the second one.

\subsection{Very ampleness}

\begin{definition}[]
A lattice polytope is \textbf{very ample} if for all vertices $v$ of $P$, the monoid
\[\ps{P\cap M-v}=\ps{m-v\mid m\in P\cap M}\]
is saturated.
\end{definition}

\begin{remark}
The idea of taking the difference with $v$ translates to making $v$ the origin

PICTURE IN THE NOTES
\end{remark}

\begin{definition}[]
If $P,Q$ are subsets of $M_\R$, their \textbf{Minkowski sum} is
\[P+Q=\cpa{p+q\mid p\in P,\ q\in Q}\]
\end{definition}
\begin{remark}
If $P=\Conv(A)$ and $Q=\Conv(B)$ then $P+Q=\Conv(A+B)$.
\end{remark}

\begin{notation}
If $k>0$ and $P=\Conv(A)$, then we set $kP$ to be the polytope defined by $\Cone(\cpa{ka\mid a\in A})$.
If $k\in \N$, this also coincides with the iterated Minkowski sum
\[\under{k\text{ times}}{P+\cdots+P}=\cpa{m_1+m_2+\cdots+m_k\mid m_i\in P}.\]
\end{notation}

\begin{remark}
If $P$ is defined by $\cpa{\ps{m,n_i}\geq b_i\mid i\in\cpa{1,\cdots, s}}$ then
\[kP=\cpa{\ps{m,n_i}\geq kb_i\ \forall i}\]
\end{remark}

\begin{fact}
Let $P\subseteq M_\R$ be a full-dimensional lattice polytope with $\rnk M\geq 2$. Then $kP$ is very ample for all $k\geq n-1$.
\end{fact}
\begin{remark}
If $\rnk M=1$ we have no issue in finding a very ample multiple.
\end{remark}

\subsection{The projective variety}
Let $P$ be a full-dimensional lattice polytope. The associated projective variety is
\[X_P=X_{(kP)\cap M}\]
for $k\in\N$ such that $kP$ is very ample.

\begin{remark}
This will yield a well defined abstract variety, though the embedding in the ambient projective spaces change with respect to $k$.
\end{remark}


Recall that $X_A\subseteq \Pj^{s-1}$ is covered by affine toric varieties: via (\ref{PrMonoidOfAffineChartOfProjectiveToricVariety}) we have (for $A=\cpa{a_1,\cdots, a_s}$)
\[X_A=\bigcup_{i=1}^s X_A\cap U_i\]
\begin{lemma}\label{LmWeOnlyNeedAffineChartsOfVertices}
If $A=P\cap M$ then
\[X_A=\bigcup_{a_i\text{ vertex of $P$}}X_A\cap U_i\]
\end{lemma}
\begin{proof}
Let $\cpa{a_j}_{j\in J}$ be the vertices of $P$. Fix $a_i\in A\bs \cpa{a_j}_{j\in J}$. We want to find $j\in J$ such that $X_A\cap U_i\subseteq X_A\cap U_j$.
Note that (exercise)
\[P\cap M_\Q=\cpa{\sum_{j\in J}r_j a_j\mid r_j\in \Q_{\geq 0},\ \sim r_j=1},\]
so we can write
\[a_i=\sum_{j\in J}r_j a_j.\]
If we clear the denominators we get
\[ka_i=\sum k_j a_j,\quad k, k_j\in \N,\ k\neq 0,\ \sum k_j=k.\]
From this we get
\[\sum_{j\in J}k_j(a_j-a_i)=0.\]
Let $j_0\in J$ be such that $k_{j_0}\neq 0$. It follows that
\[k_{j_0}(a_i-a_{j_0})=\sum_{j\in J\bs\cpa{j_0}}k_j(a_j-a_i)\]
so $a_i-a_{j_0}\in S_i=\ps{a_k-a_i\mid k\neq i}$ and $S_i$ is the monoid which corresponds to $X_A\cap U_i$. Note that $a_{j_0}-a_i\in S_i$ by definition, so also having $k_{j_0}(a_i-a_{j_0})\in S_i$ means that $a_{j_0}-a_i$ is invertible in $S_i$.

Note that $k[S_i]_{t^{a_{j}-a_i}}$ is the coordinate ring of $X_A\cap U_i\cap U_j$, but for $j_0$
\[k[S_i]_{t^{a_{j_0}-a_i}}=k[S_i]\]
so $X_A\cap U_i\cap U_{j_0}=X_A\cap U_i$, that is, $X_A\cap U_i\subseteq X_A\cap X_{j_0}$.
\end{proof}


\begin{theorem}[]\label{ThAffineChartsOfProjectiveToricVarietyFromVeryAmplePolytope}
Assume $P$ is a very ample full-dimensional lattice polytope. Then
\begin{itemize}
\item if $a_i\in P\cap M$ is a vertex, then $X_{P\cap M}\cap U_i\cong U_{\sigma_i}=\Spec k[\sigma_i^\vee\cap M]$ where $\sigma_i\subseteq N_\R$ is the strongly convex cone which is dual to $C_i=\Cone(P\cap M-a_i)$. Moreover $\dim\sigma_i=n$
\item The torus of $X_{P\cap M}$ is $T_N$.
\end{itemize}
\end{theorem}
\begin{proof}
Since $a_i$ is a vertex and $P$ is full-dimensional, $C_i$ is strongly convex and full-dimensional. Now $S_i$ (monoid that corresponds to $X_A\cap U_i$) is a submonoid $S_i\subseteq C_i\cap M=\sigma_i^\vee\cap M$ by construction.

Since $P$ is very ample, $S_i$ is saturated and as in a proof which we have seen ($2\implies 3$ from (\ref{PrCriteriaForNormalAffineToricVariety})) it follows that we have equality.

The fact that the torus is $T_N$ follows from the fact that the $\sigma_i$ are strictly convex and that the torus of $X_A$ is the same as the torus of $X_A\cap U_i$ for any $i$.
\end{proof}


The cones $\sigma_i$ assemble into the \textbf{normal fan} of the polytope $P$:
if we write \[P=\cpa{m\in M_\R\mid \ps{m,u_F}\geq -a_F\ \forall \text{$F$ facet}}\]
and fix a vertex $v\in P$, at $v$ we have a cone
\[C_v=\Cone(P\cap M-v)\]
and $\sigma_v=C_v^\vee$ as in the proof.
There is a bijection
\[\correspDef{\cpa{Q\leq P,\ v\in Q}}{\cpa{\tau\leq C_v}}{Q}{Q_v=\Cone(Q\cap M-v)}{Q_\tau=(\tau+v)\cap P}{\tau}\]
This bijection preserves inclusions, intersection, dimension etc.

PICTURE

In particular facets of $C_v$ correspond to facets of $P$ containing $v$, so
\[C_v=\cpa{m\in M_\R\mid \ps{m,u_F}\geq 0\ \forall \text{facet containing $v$}}.\]
So $\sigma_v=C_v^\vee=\Cone(u_F\mid v\in F)$.

We can extend this association $vertices\to cones$ to all faces of $P$ as follows:
\[Q\leq P\mapsto \sigma_Q=\Cone(u_F\mid Q\subseteq F)\]

\begin{example}
If $F\leq P$ is a facet, $\sigma_F$ is the ray generated by $u_F$. If $Q=P$ then $\sigma_P=\Cone(\emptyset)=\cpa{0}$.
\end{example}


\begin{definition}[]
The cones $\cpa{\sigma_Q\mid Q\leq P}$ give the \textbf{normal fan} of $P$, denoted $\Sigma_P$.
\end{definition}

\begin{definition}[]
A \textbf{fan} $\Sigma$ in $N_\R$ is a finite collection of strongly convex cones such that
\begin{enumerate}
\item for all $\sigma,\sigma'\in \Sigma$, $\sigma\cap \sigma'$ is a face of both
\item if $\sigma\in \Sigma$ and $\tau\leq \sigma$ then $\tau \in \Sigma$.
\end{enumerate}
\end{definition}


\begin{example}
PICTURE
\end{example}


\begin{proposition}[]\label{PrDualityOfFacesOfACone}
If $\tau\leq \sigma$ there is a dual face $\tau^\ast\leq \sigma^\vee$ defined as $\sigma^\vee\cap ((\Span_\R\tau)^\perp)$. This construction gives an inclusion-reversing bijection between faces of $\sigma$ and faces of $\sigma^\vee$.
\end{proposition}
\begin{example}
DRAWING FROM LECTURES
\end{example}


\begin{remark}
For all $u\in N_\R\nz$ there exists a unique $b\in \R$ such that $H_{u,b}^+\supseteq P$ and $H_{u,b}\cap P\neq \emptyset$.
\end{remark}

\begin{theorem}[]\label{ThNormalFanOfPolytopeIsAFan}
The normal fan of a polytope $P$ is a fan.
\end{theorem}
\begin{proof}[Sketch]
We have the following steps:
\setlength{\leftmargini}{0cm}
\begin{enumerate}
\item Note that
\[\sigma_Q=\cpa{u\in N_\R\mid \exists b\in \R\ s.t.\ H_{u,b}\text{ is supporting and }Q\subseteq H_{u,b}\cap P},\]
indeed
\setlength{\leftmargini}{0cm}
\begin{itemize}
\item[$\boxed{\subseteq}$] take $u\in \sigma_Q$, then $u=\sum_{Q\subseteq F}\la_F u_F$ for $\la_F\geq 0$. Let $b_0=\sum_{F\text{ facet}, Q\subseteq F}-\la_F a_F\in \R$. By construction\footnote{$\ps{m,u}=\sum \la_F\ps{m,u_F}\geq -\sum \la_F a_F=b_0$}, $P\subseteq H_{u,b_0}^+$ and $Q\subseteq H_{u,b_0}\cap P$ because $Q=\bigcap_{Q\subseteq F} F$
\item[$\boxed{\supseteq}$] Assume that $b\in \R$ is such that $H_{u,b}$ is supporting and $Q\subseteq H_{u,b}\cap P$. Let $v$ be a vertex of $Q$ (which is also a vertex of $P$). From $P\subseteq H_{u,b}^+$ and $P\in H_{u,b}$ it follows that $C_v\subseteq H_{u,0}^+$, i.e. $u\in (C_v)^\vee=\sigma_v=\Cone(u_F\mid v\in F)$, thus $u=\sum_{v\in F} \la_F u_F$ with some $\la_F\geq 0$. We have to show that if $Q\not\subseteq F$ then $\la_F=0$: fix $F_0$ such that $Q\not\subseteq F_0$ and $p\in Q\bs F_0$. $p,v\in Q\subseteq H_{u,b}$, so
\[b=\ps{p,u}=\sum \la_F\ps{p,u_F}\]
but also
\[b=\ps{v,u}=\sum \la_F\ps{v,u_F}=-\sum_{v\in F}\la_F a_F\]
so $\sum_{v\in F}\la_F\ps{p,u_F}=-\sum_{v\in F}\la_F a_F$, but $\ps{p,u_F}\geq -a_F$ for all $F$, so we get equality everywhere $\la_F\neq 0$. Since $p\notin F_0$ we have $\ps{p,u_{F_0}}>-a_{F_0}$, so $\la_{F_0}=0$.
\end{itemize}
\setlength{\leftmargini}{0.5cm}
\item If $Q\leq P$ and $F\leq P$ facet then $u_F\in \sigma_Q$ if and only if $Q\subseteq F$
\item if $Q\subseteq Q'$ then $\sigma_{Q'}\leq \sigma_{Q}$ and all faces of $\sigma_Q$ are of this form\footnote{for this you need duality of faces for a cone $\sigma$ (\ref{PrDualityOfFacesOfACone}).}.
\item $\sigma_Q\cap \sigma_{Q'}=\sigma_{Q''}$ where $Q''$ is the smallest face of $P$ which contains both $Q$ and $Q'$.
\end{enumerate}
\setlength{\leftmargini}{0.5cm}
\end{proof}


\begin{remark}
$\sigma_Q$ is strictly convex because each $\sigma_Q$ is a face of some $\sigma_v$ and $\sigma_v$ is strictly convex because $P$ is full-dimensional.
\end{remark}

\begin{remark}
The $\sigma_v$ are the \textbf{maximal cones} of $\Sigma_P$ since any other $\sigma_Q$ is a face of some $\sigma_v$.
\end{remark}


\begin{definition}[]
A fan $\Sigma$ in $N_\R$ is called \textbf{complete} if
\[\abs{\Sigma}=\bigcup_{\sigma\in \Sigma}=N_R\]
\end{definition}



\begin{proposition}[]
If $P$ is a full-dimensional lattice polytope then $\Sigma_P$ is complete.
\end{proposition}
\begin{proof}
Fix $u\in N_\R\nz$ and set $b=\min\cpa{\ps{v,u}\mid v\text{ vertex of $P$}}$. Then $P\subseteq H_{u,b}^+$ and there exists $v_0$ vertex such that $\ps{v_0,u}=b$, that is, $v_0\in H_{u,b_0}$. From what we have seen, this implies that $u\in\sigma_{v_0}\subseteq \abs{\Sigma_P}$.
\end{proof}

\begin{remark}
The normal fan of $P$ is invariant with respect to dilations and translations by integral vectors, that is, 
\[\Sigma_P=\sigma_{kP+m}\]
for any $k\in \N$ and $m\in M$.
\end{remark}

Together with the next proposition, this implies that the projective toric varieties $X_{kP}, X_P,\ X_{P+m}$ are all abstractly isomorphic. The only difference is the embedding in projective space.


\begin{proposition}[]\label{PrIntersectionOfAffineChartsForProjectiveToric}
If $P$ is a very ample full-dimensional lattice polytope. Let $v\neq w$ be vertices of $P$ and let $Q$ be the smallest face of $P$ which contains both. Then\footnote{$\sigma_Q=\sigma_v\cap \sigma_w$ so intersections at the level of cones in the fan describe how the affine patches of the toric variety are glued together.}
\[X_{P\cap M}\cap U_v\cap U_w\cong U_{\sigma_Q}=\Spec k[\sigma_Q^\vee\cap M].\]
\end{proposition}
\begin{proof}
We have inclusions
% https://q.uiver.app/#q=WzAsNSxbMSwyLCJYX3tQXFxjYXAgTX1cXGNhcCBVX3ZcXGNhcCBVX3ciXSxbMCwxLCJYX3tQXFxjYXAgTX1cXGNhcCBVX3YiXSxbMCwwLCJVX3tcXHNpZ21hX3Z9Il0sWzIsMSwiWF97UFxcY2FwIE19XFxjYXAgVV93Il0sWzIsMCwiVV97XFxzaWdtYV93fSJdLFsyLDEsIj0iLDMseyJzdHlsZSI6eyJib2R5Ijp7Im5hbWUiOiJub25lIn0sImhlYWQiOnsibmFtZSI6Im5vbmUifX19XSxbMSwwLCJcXHN1cHNldGVxIiwzLHsic3R5bGUiOnsiYm9keSI6eyJuYW1lIjoibm9uZSJ9LCJoZWFkIjp7Im5hbWUiOiJub25lIn19fV0sWzQsMywiPSIsMyx7InN0eWxlIjp7ImJvZHkiOnsibmFtZSI6Im5vbmUifSwiaGVhZCI6eyJuYW1lIjoibm9uZSJ9fX1dLFswLDMsIlxcc3Vic2V0ZXEiLDMseyJzdHlsZSI6eyJib2R5Ijp7Im5hbWUiOiJub25lIn0sImhlYWQiOnsibmFtZSI6Im5vbmUifX19XV0=
\[\begin{tikzcd}
	{U_{\sigma_v}} && {U_{\sigma_w}} \\
	{X_{P\cap M}\cap U_v} && {X_{P\cap M}\cap U_w} \\
	& {X_{P\cap M}\cap U_v\cap U_w}
	\arrow["{=}"{marking, allow upside down}, draw=none, from=1-1, to=2-1]
	\arrow["{=}"{marking, allow upside down}, draw=none, from=1-3, to=2-3]
	\arrow["\supseteq"{marking, allow upside down}, draw=none, from=2-1, to=3-2]
	\arrow["\subseteq"{marking, allow upside down}, draw=none, from=3-2, to=2-3]
\end{tikzcd}\]
and we identify the double intersection both with $(U_{\sigma_v})_{t^{w-v}}\subseteq U_{\sigma_v}$ and $(U_{\sigma_w})_{t^{v-w}}\subseteq U_{\sigma_w}$. 

We need to show that, for instance, $(U_{\sigma_v})_{t^{w-v}}$ can be identified with $U_{\sigma_{Q}}$. Note that $w-v\in C_v=\sigma_v^\vee$ so $\tau:=H_{w-v}\cap \sigma_v\leq \sigma_v$. We saw that $(U_{\sigma_v})_{t^{w-v}}\cong U_\tau$ (\ref{ThAffineChartsOfProjectiveToricVarietyFromVeryAmplePolytope}).


Let us check that $\tau=\sigma_Q$. We know that $\sigma_Q=\sigma_v\cap \sigma_w$ from the proof that the normal fan is a fan (\ref{ThNormalFanOfPolytopeIsAFan}), i.e. we want $H_{w-v}\cap \sigma_v=\sigma_w\cap \sigma_v$.
\setlength{\leftmargini}{0cm}
\begin{itemize}
\item[$\boxed{\subseteq}$] If $n\in H_{w-v}\cap \sigma_v\nz$ then there exists a unique $b\in\R$ such that $H_{u,b}$ is supporting for $P$ and $u\in \sigma_v$ implies $v\in H_{u,b}$ (proposition from little ago). Also $u\in H_{w-v}$, that is, $\sp{w,u}=\ps{v,u}$, so putting the two facts together $\ps{w,u}=b$, that is, $u\in \sigma_w$.
\item[$\boxed{\supseteq}$] If $u\in \sigma_v\cap \sigma_w\nz$ and $b\in \R$ such that $H_{u,b}$ supporting then $u\in \sigma_v$ implies $v\in H_{u,b}$ and so .................... $\ps{w-v,u}=0$ which implies $u\in H_{w-v}$.
\end{itemize}
\setlength{\leftmargini}{0.5cm}
\end{proof}

\begin{remark}
What we are sayng is that the toric variety depends only on the fan in some sense, not the polytope.
\end{remark}



\begin{remark}
This shows that for a full-dimensional lattice polytope $P$, $X_P=X_{(kP)\cap M}$ where $kP$ is very ample as an abstract variety / scheme only depends on the normal fan $\Sigma_P$ and can be constructed directly from it.
\end{remark}


\begin{example}
Let $P=\Delta_n=\Cone(0,e_1,\cdots, e_n)\subseteq \R^n$. Let $A=\Delta_n\cap \Z^n=\cpa{0,e_1,\cdots, e_n}$.
\[\phi_A:\funcDef{\G_m^n}{\Pj^n}{(a_1,\cdots, a_n)}{[1,a_1,a_2,\cdots, a_n]}\]
and this is exactly an embedding of the torus of $\Pj^n$, which is dense, so $X_{\Delta_n}=\Pj^n$.


Let us now try $k\Delta_n$. Then $X_{k\Delta_n}$ is still isomorphic to $\Pj^n$ but it is embedded in $\Pj^{\binom{n+k}k-1}$ via the Veronese embedding. For example, for $n=k=2$ we have $2\Delta_2\cap \Z^2=\cpa{(0,0),(1,0),(2,0),(0,1),(0,2),(1,1)}$ and
\[\phi_A:\funcDef{\G_m^2}{\Pj^5}{(a,b)}{[1,a,a^2,b,b^2,ab]}\]
This extends to
\[\funcDef{\Pj^2}{\Pj^5}{[x_0,x_1,x_2]}{[x_0^2,x_0x_1,x_1^2,x_0x_2,x_2^2,x_1x_2]}\]
which is the Veronese embedding.
\end{example}


\begin{example}
Consider the trapezoids given by the convex hull of 
\[(0,1), (0,0), (0,b), (0,a)\]
and let $X_{a,b}$ be the associated toric variety. If $b-a=b'-a'$ then $X_{a,b}\cong x_{a',b'}$ because the fan doesn't change (even though we it's not necessarily the case that we get between such isomorphic polytope by scaling and translating).

This toric variety is called Hirzebruch surface $H_r$ where $r=b-a\in \N$. Another description for it is $\Pj(\Oc\oplus \Oc(-r))$.
\end{example}





\begin{proposition}[]\label{PrSmoothnessAndNormalityOfProjectiveToricVarieties}
If $P$ is a full-dimensional lattice polytope, then $X_P$ is normal (because the affine pieces are of the form $U_{\sigma_v}$ for $\sigma_v$ stictly convex) and $X_P$ is smooth if and only if $\Sigma_P$ is smooth fan (i.e. all cones in $\Sigma_P$ are smooth).
\end{proposition}
\begin{proof}
It follows from previous results and locality of the two properties.
\end{proof}