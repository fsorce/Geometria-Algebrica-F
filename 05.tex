\chapter{Divisors on toric varieties}

\section{Class group}
Let $\Sigma$ be a fan in $N_\R$. 
\begin{notation}
Set $\Sigma(k):=\cpa{\sigma\in \Sigma\mid \dim \sigma=k}$.
\end{notation}

For each $\rho\in \Sigma(1)$, by orbit-cone correspondence (\ref{PrOrbit-ConeCorrespondence}), we get a prime divisor $D_\rho=V(\rho)=\ol{\Oc(\rho)}$.

Let $v_\rho:k(X_\Sigma)^\ast\to \Z$ be the valuation of the local ring of $D_\rho$.


Recall that each $\rho$ has a minimal ray generator $u_\rho\in \rho\cap N$. Note that for $m\in M$, 
\[\chi^m:T_N\to \G_m\]
is a regular function defined on an open subset of $X_\Sigma$, so it yields a rational function on $X_\Sigma$.

\begin{proposition}[]\label{PrValuationOfCharacter}
$v_\rho(\chi^m)=\ps{m,u_\rho}$.
\end{proposition}
\begin{proof}
Extend $u_\rho$ to a $\Z$-basis of $N$, say $u_\rho,e_2,\cdots, e_n$.
Recall that $U_\rho\cong \A^1\times\G_m^{n-1}$ and $D_\rho\cap U_\rho$ under the isomorphism is described by $x_1=0$, so \[\Oc_{X_\Sigma,D_\rho}\cong \Oc_{U_\rho,D_\rho\cap U_\rho}\cong \Oc_{\A^1\times\G_m^{n-1},\cpa{x_1=0}}= k[x_1,x_2^{\pm},\cdots, x_n^{\pm}]_{(x_1)}.\]
Thus $v_\rho(f)$ is the unique integer such that $f=x_1^{v_\rho(f)}g/h$ where $g,h\notin (x_1)$ for any $f\in k(X_\Sigma)^\ast$.

Using the dual basis to the $u_\rho,e_2,\cdots, e_n$ of $N$ we can write $m=\sum \ps{m,e_i}e_i^\vee$ (we set $e_1=u_\rho$), so
\[\chi^m=\prod \chi^{\ps{m,e_i}e_i^\vee}=x_1^{\ps{m,u_\rho}}\cdot x_2^{\ps{m,e_2}}\cdots x_n^{\ps{m,e_n}}.\]
\end{proof}


\begin{proposition}[]
We have that
\[\div(\chi^m)=\sum_{\rho\in\Sigma(1)}\ps{m,u_\rho}D_\rho\]
\end{proposition}
\begin{proof}
The irreducible components of $X_\Sigma\bs T_N$ are exactly the $D_\rho$ by the orbit-cone correspondence. Moreover, $\chi^m$ is defined and not zero on $T_N$ so it follows that $\Supp(\div(\chi^m))\subseteq X_\Sigma\bs T_N$. By the previous computation we are done.
\end{proof}



\begin{remark}
Note that $D_\rho$ are torus invariant and so every linear combination $\sum a_\rho D_\rho$ is torus-invariant (with the obvious induced action).

These are actually ALL the torus-invariant divisors (easy from the orbit-cone correspondence).
\end{remark}


\begin{notation}
We use $\Div_{T_N}(X_\Sigma)$ to denote the subgroup of $\Div(X_\Sigma)$ given by the torus-invariant divisors.
\end{notation}


\begin{proposition}[]\label{PrExactSequenceOfTorusInvariantDivisors}
There is an exact sequence
\[M\to \Div_{T_N}(X_\Sigma)\to \Cl(X_\Sigma)\to 0\]
where the first map is $m\mapsto \div(\chi^m)$. In particular every divisor on $X_\Sigma$ is linearly equivalent to a torus-invariant one.

Moreover, the first map is injective if and only if $\cpa{u_\rho\mid \rho\in \Sigma(1)}$ spans $N_\R$.
\end{proposition}
\begin{proof}
We start from the localization sequence induced by the inclusion $T_N\subseteq X_\Sigma$ (we are using the fact that the $D_\rho$ generate the invariant divisors)
\[\Div_{T_N}(X_\Sigma)\to \Cl(X_\Sigma)\to \Cl(T_N)\to 0\]
Since $k[x_1^{\pm},\cdots, x_n^{\pm}]$ is a UFD we have that $\Cl(T_N)=0$.

The composite $M\to \Div_{T_N}(X_\Sigma)\to \Cl(X_\Sigma)$ is clearly 0 because the images of the first map are principal.

Suppose $D\in \Div_{T_N}(X_\Sigma)$ is such that $D=\div(f)=\sum a_\rho D_\rho$. Then $\div(f)\res{T_N}=0$, so $f\in \Oc_{T_N}^\ast$, that is, $f=c\chi^m$. So $\div(f)=\div(c\chi^m)=\div(\chi^m)$, proving exactness.


Now, $\div(\chi^m)=0$ in $\Div_{T_N}(X_\Sigma)$ means that $\ps{m,u_\rho}=0$ for all $\rho\in \Sigma(1)$. This is equivalent to $m=0$ exactly when the $\cpa{u_\rho}$ span $N_\R$.
\end{proof}

\begin{remark}
The condition ``$\cpa{u_\rho}$ spans $N_\R$" corresponds geometrically to the fact that $X_\Sigma$ has no ``torus factors", that is, $X_\Sigma$ is not equivariantly isomorphic to some $\G_m^k\times X_{\Sigma'}$.
\end{remark}

\begin{proposition}[]
The following are equivalent
\begin{enumerate}
\item $X_\Sigma$ has a torus factor
\item there exists a non-constant morphism $X_\Sigma\to \G_m$
\item The $\cpa{u_\rho\mid \rho\in \Sigma(1)}$ do not span $N_\R$.
\end{enumerate}
\end{proposition}
\begin{proof}
We give the implications
\setlength{\leftmargini}{0cm}
\begin{itemize}
\item[$\boxed{1.\implies2.}$] If $X_\Sigma\cong \G_m^k\times X_{\Sigma'}$ we can take any projection on one of the first $k$ factors. 
\item[$\boxed{2.\implies3.}$] If $f:X_\Sigma\to \G_m$ is non-constant then the restriction $f\res{T_N}:T_N\to \G_m$ is a non-constant morphism, i.e. a non-constant invertible element of $\Oc_{T_N}(T_N)$, so $f\res{T_N}=c\chi^m$ for $c\in k^\ast$ and $m\in M\nz$.

By multiplying by $c\ii$ we can assume that $c=1$, so $f:X_\Sigma\to \G_m$ is now a toric morphism, since it restricts to a homomorphism on the tori. These are induced by a non-constant homomorphism $\vp:N\to \Z$ which is compatible with the fans (\ref{ThCharacterizationOfToricMorphisms}). The fan of $\G_m$ is the origin inside $\R=\Z_\R$, so $\vp(\abs{\Sigma})=0$ and so $\cpa{u_\rho\mid \rho\in \Sigma(1)}$ does not span $N_\R$, otherwise $\vp:N\to \Z$ would have to be the $0$ map.
\item[$\boxed{3.\implies1.}$] Basically already seen. You get a proper sublattice of $N$ by taking the $\Z$-span of $\cpa{u_\rho}$ and now so complete to a basis.
\end{itemize}
\setlength{\leftmargini}{0.5cm}
\end{proof}




We can now compute $\Cl(X_\Sigma)$ algorithmically:
\begin{enumerate}
\item Fix a basis $e_1,\cdots, e_n$ for $M$
\item Fix minimal ray generators $u_1,\cdots, u_r$ which form a basis for $\Div_{T_N}(X_\Sigma)$ ******
\item If $\cpa{\rho_1,\cdots,}$
\item % https://q.uiver.app/#q=WzAsNCxbMCwwLCJNIl0sWzEsMCwiXFxEaXZfe1RfTn0oWF9cXFNpZ21hKSJdLFswLDEsIlxcWl5uIl0sWzEsMSwiXFxaXnIiXSxbMCwxXSxbMiwzXSxbMCwyLCJcXGNvbmciLDMseyJzdHlsZSI6eyJib2R5Ijp7Im5hbWUiOiJub25lIn0sImhlYWQiOnsibmFtZSI6Im5vbmUifX19XSxbMSwzLCJcXGNvbmciLDMseyJzdHlsZSI6eyJib2R5Ijp7Im5hbWUiOiJub25lIn0sImhlYWQiOnsibmFtZSI6Im5vbmUifX19XV0=
\[\begin{tikzcd}
	M & {\Div_{T_N}(X_\Sigma)} \\
	{\Z^n} & {\Z^r}
	\arrow[from=1-1, to=1-2]
	\arrow["\cong"{marking, allow upside down}, draw=none, from=1-1, to=2-1]
	\arrow["\cong"{marking, allow upside down}, draw=none, from=1-2, to=2-2]
	\arrow[from=2-1, to=2-2]
\end{tikzcd}\]
where the second isomorphism is using the basis $D_{\rho}$. This map corresponds to the matrix
\[\mat{\ps{e_1,u_1} &\cdots &\ps{e_n,u_1}\\\vdots&\ddots&\vdots\\\ps{e_1,u_r}&\cdots&\ps{e_n,u_r}}\]
and you compute the cokernel of this matrix using the smith normal form.
\end{enumerate}



\begin{remark}
Using this description, you can show that $\Cl(X_\Sigma\times X_{\Sigma'})=\Cl(X_\Sigma)\oplus\Cl(X_{\Sigma'})$, which is NOT true for general normal varieties.
\end{remark}




\section{Cartier divisors on toric varieties}

To get $\Pic(X_\Sigma)$ we want to consider $\CDiv_{T_N}(X_\Sigma)\subseteq \Div_{T_N}(X_\Sigma)$. Note that $M\to \Div_{T_N}(X_\Sigma)$ has image contained in $\CDiv_{T_N}(X_\Sigma)$.


\begin{proposition}[]\label{PrExactSequenceOfTorusInvariantCartierDivisors}
We have an exact sequence
\[M\to \CDiv_{T_N}(X_\Sigma)\to \Pic(X_\Sigma)\to 0\]
This is also exact on the left exactly when $X_\Sigma$ has no torus factors
\end{proposition}


\subsection{Torus invariant Cartier divisors: affine case}
What is $\CDiv_{T_N}(X_\Sigma)$? For Weil divisors we had $\Div_{T_N}(X_\Sigma)=\bigoplus \Z D_\rho$.

\begin{proposition}[]\label{PrInvariantCartierDivisorsOnAffineArePrincipalAndComeFromCharacters}
Let $\sigma\subseteq N_\R$ be a strongly convex cone. Then every $T_N$-invariant Cartier divisor on $U_\sigma$ is the divisor of a character, so $\Pic(U_\sigma)=0$.
\end{proposition}
\begin{proof}
Let $R=k[S_\sigma]$. Suppose that $D=\sum a_\rho D_\rho$ is $T_N$-invariant and Cartier. Suppose $a_\rho\neq 0$ for the $\rho$ which appear in the sum.

Suppose $D$ is effective. Note that the $D_\rho$ intersect non-trivially: $\Oc(\sigma)\subseteq \ol{\Oc(\rho)}=D_\rho$, so $\Oc(\sigma)\subseteq \bigcap D_\rho$. Pick any point $p\in \Oc(\sigma)$. Since $D$ is Cartier, there exists $U\subseteq U_\sigma$ open such that $p\in U$ and $D\res U=\div(f)\res U$ for some $f\in k(U_\sigma)^\ast$. We can assume $U=(U_\sigma)_h=\Spec R_h$ for some $h\in R$ because principal open sets form a basis. Since $D$ is effective, $\div(f)\res U\geq 0$, so $f\in R_h$. If $f=g/h^k$, since $h\in R_h^\ast$ and multiplying by $h$ doesn't change $\div(f)\res U$, we can just assume $f=g\in R$.

Consider now\footnote{this set is $\Gamma(\Oc(-D))$} $I=\cpa{g\in k(U_\sigma)\mid g=0\text{ or }\div g\geq D}$. Since $D\geq 0$ we have $\div g\geq 0$ so $I$ is an ideal of $R$. This ideal is $T_N$-invariant because $D$ is. Thherefore $I$ is a subrepresentation of $R$ under the action of $T_N$, meaning that 
\[I=\bigoplus_{\div (t^m)\geq D} kt^m\]
Note that $f\in I$ because
\[\div f=\sum v_\rho(f)D_\rho+\under{\geq 0\text{ because }f\in R}{\sum_{E\neq D_\rho}v_E(f)E}\geq \sum v_\rho(f)D_\rho+0=D\]
where the last equality holds because of our choice for $U$.

Thus $f=\sum a_{i}t^{m_i}$ with $\div(t^{m_i})\geq D$ for all $i$. On $U$, 
\[\div(t^{m_i})\res U\geq D\res U=\div(f)\res U\implies \div(t^{m_i}/f)\res U\geq 0\implies t^{m_i}/f\in \Oc(U)\]
So we can write
\[1=\sum a_i t^{m_i}/f\]
and the fact that $t^{m_i}/f\in \Oc(U)$ we can evaluate that equality at $p\in U$ to get
\[1=\sum a_i t^{m_i}(p)/f(p)\]
so $t^{m_i}/f(p) \neq 0$ for some $i$. Let $V\subseteq U$ be an open set where $t^{m_i}/f$ is never $0$ and $p\in V$. We get
\[\div(t^{m_i}/f)\res V=0\implies \div(t^{m_i})\res V=\div (f)\res V=D\res V\]
Since $V$ intersects every $D_\rho$ non-trivially we get that $\div(t^{m_i})=D$ on $U_\sigma$.

\bigskip

Consider now a general invariant Cartier divisor $D$. Since $\sigma$ is strongly convex, there exists some $m\in M$ such that $\ps{m,u_\rho}>0$ for all $\rho\in \sigma(1)$ (because the origin is a face of $\sigma$). This implies that $D'=D+\div(t^{km})$ for $k$ big enough we get an effective invariant Cartier divisor. By the previous case $D'=\div(t^{m_i})$ and so $D=\div(t^{m_i-km})$.
\end{proof}



\begin{example}
If $\sigma=\Cone((d,-1),(0,1))\subseteq \R^2$ we have that $\Cl(U_\sigma)\cong \znz d$ but $\Pic(U_\sigma)=0$, so the generator of the class group corresponds to a Weil divisor which is not Cartier.

If we take out the torus fixed point $\gamma_\sigma\in U_\sigma$, $X=U_\sigma\bs\cpa{\gamma_\sigma}$ (we remove the maximal cone). Now $X$ is a smooth toric variety and in this case $\Pic(X)=\Cl(X)$. We left the rays untouched so $\Cl(X)=\Cl(U_\sigma)$, so we suddenly got nontrivial line bundles by removing a point!
\end{example}

\begin{remark}
You can show that if $\Sigma$ contains a cone of dimension equal to that of the lattice then $\Pic(X_\Sigma)$ is torsion free.
\end{remark}
\begin{proof}[Sketch]
Suppose $\Pic(X_\Sigma)$ has torsion and take a representative in $\CDiv_{T_N(X_\Sigma)}$, that is, take an invariant Cartier Divisor $D$ such that $kD=\div(t^m)$. We have to show that $D=\div(t^{m'})$.

If $\sigma$ is a cone of maximal dimension then $\cpa{u_\rho\mid \rho\in \sigma(1)}$ spans $N_\R$. $D\res{U_\sigma}$ is the divisor of a character $t^{m'}$ by (\ref{PrInvariantCartierDivisorsOnAffineArePrincipalAndComeFromCharacters}).

Since $\div(t^m)\res{U_\sigma}=\div(t^{km'})\res{U_\sigma}$ then $m=km'$ and so we get $D=\div(t^{m'})$ on $X_\Sigma$ because the rays of $\sigma$ span (details to fill in).
\end{proof}


\begin{proposition}[]\label{PrSmoothnessOfToricVarietyIsEquivalentToPicardBeingTheClassGroup}
$X_\Sigma$ is smooth if and only if $\Pic(X_\Sigma)=\Cl(X_\Sigma)$
\end{proposition}
\begin{proof}
The first implication is always true. Suppose $\Pic(X_\Sigma)=\Cl(X_\Sigma)$. We want to show that $\sigma\in \Sigma$ is smooth.

Since $\Cl(X_\Sigma)\to \Cl(U_\sigma)$ is surjective, every divisor on $U_\sigma$ is Cartier because the restriction of a Cartier divisor is Cartier and $\Cl(X_\Sigma)=\Pic(X_\Sigma)$. Since $\Pic(U_\sigma)=0$ it follows that $\Cl(U_\sigma)=0$, thus $M\to \Div_{T_N}(U_\sigma)$ is surjective.

If $\sigma(1)=\cpa{\rho_1,\cdots, \rho_r}$ then the map $M\to \Div_{T_N}(U_\sigma)$ in coordinate is
\[\funcDef{M}{\Z^r}{m}{(\ps{m,u_{\rho_i}})}\]
This map is dual to
\[\phi:\funcDef{\Z^r}{N}{e_i}{u_{\rho_i}}\]
Since $\phi^\vee$ is surjective it follows that (using a little homological algebra) $\phi$ is injective and that $\coker\phi$ is torsion-free. This implies that $\cpa{u_{\rho_i}}$ can be completed to a $\Z$-basis of $N$ and so $\sigma$ is a smooth cone.
\end{proof}


\begin{remark}
One can also show that $X_\Sigma$ is simplicial (every cone in $\Sigma$ is simplicial, that is, the minimal ray generators are $\R$-linearly independent) if and only if every Weil divisor is $\Q$-Cartier, that is, it has a positive multiple which is Cartier, that is, the index $[\Cl(X_\Sigma):\Pic(X_\Sigma)]$ is finite. 
\end{remark}



\subsection{Torus invariant Cartier divisors}

Recall that if $D\in \Div(X)$ is Cartier then there exists an open cover $\cpa{U_i}$ of $X$ and $f_i\in k(X)$ for all $i$ such that $D\res{U_i}=\div(f_i)\res{U_i}$. We may call $\cpa{(U_i,f_i)}$ a \textit{Cartier data} for $D$.

\begin{proposition}[]\label{PrInvariantCartierDivisorsCorrespondToCartierData}
Let $D\in \Div_{T_N}(X_\Sigma)$ and let us write $D=\sum a_\rho D_\rho$. The following are equivalent
\begin{enumerate}
\item $D$ is Cartier
\item $D$ is principal on $U_\sigma$ for all $\sigma\in \Sigma$.
\item For all $\sigma\in \Sigma$ there exists $m_\sigma\in M$ such that\footnote{i.e. $\sum a_\rho D_\rho\cap U_\sigma=D\res{U_\sigma}=\div(t^{-m_\sigma})\res{U_\sigma}$ and $D_\rho\cap U_\sigma\neq \emptyset$ if and only if $\rho\leq \sigma$.} $\ps{m_\sigma,u_\rho}=-a_\rho$ for all $\rho\in \sigma(1)$.
\item for all\footnote{maximal cones in $\Sigma$ by inclusion. They may be of different dimension.} $\sigma\in \Sigma_{max}$ there exists $m_\sigma\in M$ such that $\ps{m_\sigma,u_\rho}=-a_\rho$ for all $\rho\in \sigma(1)$.
\end{enumerate}
Moreover, if $D$ is Cartier, then the $m_\sigma$ in proposition $4$ are unique in the quotient $M/M(\sigma)$ for $M(\sigma)=\sigma^\perp\cap M$.

In particular, if $\tau\leq \sigma$ then $m_\tau\equiv m_\sigma$ modulo $M(\tau)$.
\end{proposition}
\begin{proof}
The only new part for the equivalences is $4.$. $3.$ clearly implies $4.$. The fact that $4$ implies the others follows from the fact that if $\sigma\leq \sigma'$ then $m_{\sigma'}$ is a valid choice for $m_\sigma$ in proposition 3.

Now we prove the uniqueness. If $m_\sigma$ and $m_{\sigma'}$ are two elements that satisfy the condition, then $\ps{m_\sigma,u_\rho}=-a_\rho=\ps{m_{\sigma'},u_\rho}$ for all $\rho\in \sigma(1)$. Thus
\[\ps{m_\sigma-m_{\sigma'},u_\rho}=0\implies \ps{m_\sigma-m_{\sigma'},u}=0\ \forall u\in \sigma.\]
\end{proof}


\begin{notation}
We call $\cpa{m_\sigma}_{\sigma\in \Sigma}$ the \textit{Cartier data} for the divisor $D$.
\end{notation}

\begin{remark}
The minus sign in $\ps{m_\sigma,u_\rho}=-a_\rho$ is a convention. It matches the convention used when writing the presentations for polytopes $\ps{m,u_F}\geq -a_F$.

This choice boils down to the fact that the $\Oc_X$-module associated to a divisor $D$ is defined as
\[\Oc_X(D)(U)=\cpa{f\in k(X)^\ast\mid (\div f+D)\res U\geq 0}\cup\cpa{0}.\]
\end{remark}



Note that if $\tau\leq \sigma$ then we have $\sigma^\perp\cap M\subseteq \tau^\perp\cap M$ and this induces a map $M/M(\sigma)\to M/M(\tau)$. This maps $[m_\sigma]$ to $[m_\tau]$.

The groups $\cpa{M/M(\sigma)}_{\sigma\in \Sigma}$ form an inverse system ordered by $\Sigma$ and the face-relation. It turns out that
\[\CDiv_{T_N}(X_\Sigma)\cong \varprojlim_{\sigma\in \Sigma}M/M(\sigma).\]
This just means that Cartier divisors correspond exactly to compatible collections of $\cpa{[m_\sigma]}_{\sigma\in \Sigma}$.


\subsubsection{Support functions}
These $(m_\sigma)_{\sigma\in\Sigma}$ can be though of as piecewise linear functions on $\abs\Sigma$

\begin{definition}[]
A \textbf{support function} on a fan $\Sigma$ is $N_\R$ is a function $\vp:\abs{\Sigma}\to \R$ which is linear on every cone $\sigma$. Such a function is \textbf{integral} if $\vp(\abs\Sigma\cap N)\subseteq \Z$.


We use $SF(\Sigma,N)$ to denote the group (sum induced via the codomain) of integral support functions on $\Sigma$.
\end{definition}

\begin{theorem}[]\label{ThIntegralSupportFunctionCorrespondToInvariantCartierDivisors}
Given $D=\sum a_\rho D_\rho\in \CDiv_{T_N}(X_\Sigma)$ with Cartier data $(m_\sigma)_{\sigma\in \Sigma}$, we get a function
\[\vp_D:\funcDef{\abs\Sigma}{\R}{u}{\ps{m_\sigma,u}\text{ if }u\in \sigma}\]
This is a well defined integral support function on $\Sigma$.

Moreover $D=-\sum \vp_D(u_\rho)D_\rho$ and the map $D\mapsto \vp_D$ given an isomorphism between $\CDiv_{T_N}(X_\Sigma)\cong SF(\Sigma,N)$.
\end{theorem}
\begin{proof}
The fact that $\vp_D$ is well defined follows from the fact that if $u\in \sigma\cap \sigma'$ then $m_\sigma\equiv m_{\sigma'}\pmod{M(\sigma\cap \sigma')}$ and so $m_\sigma-m_{\sigma'}\in (\sigma\cap \sigma')^\perp\subseteq u^\perp$.

It is integral because $\ps{\cdot,\cdot}$ is a pairing between $N$ and $M$.

The identity $\vp_D(u_\rho)=-\ps{m_\sigma,u_\rho}$ is already known for Cartier data (\ref{PrInvariantCartierDivisorsCorrespondToCartierData}).

It is clear that $D\mapsto \vp_D$ is a homomorphism. It is injective thanks to the fact we just mentioned. Consider now $\vp\in SF(\Sigma,N)$. For a fixed $\sigma\in \Sigma$ note that $\vp\res{\sigma\cap N}:\sigma\cap N\to \Z$ is a monoid homomorphism by linearity of $\vp\res\sigma$, this induces a homomorphism $N_\sigma:=\Span(\sigma)\cap N\to \Z$, which gives an element of $(N_\sigma)^\vee\cong M/M(\sigma)$. These $m_\sigma$ yield Cartier data of a divisor and by construction $\vp_D=\vp$.
\end{proof}


We may therefore rewrite the localization sequence (\ref{PrExactSequenceOfTorusInvariantCartierDivisors}) as
\[M\to SF(\Sigma,N)\to \Pic(X_\Sigma)\to 0\]
where the first map is $m\mapsto (u\mapsto -\ps{m,u})$. The image of $M$ in $SF(\Sigma,N)$ are exactly the linear functions (seen as a subgroup of the piecewise linear ones).

\section{Proper but not projective variety}

These piecewise linear functions are useful to state some geometric properties, for example:

\begin{fact}[]
A divisor $D$ is ample if and only if $\vp_D$ is strictly convex, that is, it is convex and it is linear only on the cones of $\Sigma$, not bigger subsets.
\end{fact}

We will produce a toric 3-fold whose fan is complete (and thus is proper) but such that all support functions are linear, meaning that $\Pic(X_\Sigma)=0$ and so we have no hope of finding an ample line bundle to embed $X_\Sigma$ in projective space.



Consider the cube $[-1,1]^3\subseteq \R^3$ and take the fan over its faces. Now take one vertex (say $(1,1,1)$) and we drag it up, making it $(1,2,3)$.

This is a complete fan by construction, so $X_\Sigma$ is proper by (\ref{ThPropernessConditionForToricMorphism}).

Consider now $\vp\in SF(\Sigma,\Z^3)$. The minimal ray generators are
\[u_1=(1,2,3),\ u_2=(1,-1,1),\ u_3=(1,1,-1),u_4=(-1,1,1)\]
\[u_5=(1,-1,-1),\ u_6=(-1,-1,1),\ u_7(-1,1,-1),\ u_8=(-1,-1,-1)\]
Let $\sigma_1=\Cone(u_1,u_2,u_3,u_5)$. By definition $\vp_{\sigma_1}$ is linear, say given by $m_1\in \Z^3$. We can replace $\vp$ by $\vp-\ps{m_1,\cdot}$. We claim that now $\vp=0$ (so the original $\vp$ was the linear function $\ps{m,\cdot}$). On $\sigma_1$ we have $\vp=0$ by construction, so in particular
\[\vp(u_1)=\vp(u_2)=\vp(u_3)=\vp(u_5)=0\]
we have to show that $\vp$ is zero on the other minimal ray generators. Every cone of $\sigma$ gives a linear relation between the four corresponding ray generators:
\begin{gather*}
2u_1+5u_5=4u_2+3u_3\\
2u_1+4u_7=3u_3+5u_4\\
2u_1+3u_6=4u_2+5u_4\\
u_2+u_8=u_5+u_6\\
u_3+u_8=u_5+u_7\\
u_4+u_8=u_6+u_7
\end{gather*} 
Applying $\vp$ we get some equations between the values of $\vp$ at the ray generators, which will give the desired result after some computation.


\begin{remark}
This example is clearly singular (it's not simplicial since the cones all have four generators). One can also produce a smooth example by producing a smooth fan with no strictly convex support function.
\end{remark}

The reason why we considered 3-folds is because

\begin{fact}[]
Every proper toric surface is projective.
\end{fact}

\begin{fact}[]
Smooth proper surfaces over an algebraically closed field are always projective, but there are singular proper non-projective surfaces.
\end{fact}


\section{Projective case}
Recall that if we have a full dimensional lattice polytope in $M_\R$
\[P=\cpa{m\in M_\R\mid \ps{m,u_F}\geq -a_F\text{ for $F$ facet}}\]
The toric variety only depends on the normal fan $\Sigma_P$, the polytope itself actually encodes a divisor on $X_P$, specifically the divisor
\[D_P=\sum a_F D_{F}\]
where $D_F=D_{\Cone(u_F)}$ (recall that rays of $\Sigma_P$ correspond to facets of $P$).

\begin{proposition}[]\label{PrPolytopeGivesCartierDivisor}
$D_P$ is Cartier.
\end{proposition}
\begin{proof}
We produce Cartier data, that is, for every maximal cone $\sigma\in \Sigma_P$ we want $m_\sigma\in M$ such that $\ps{m_\sigma,u_F}=-a_F$ for every facet $F$ such that $\rho_F\in \sigma(1)$. Recall that maximal cones in $\Sigma_P$ correspond to vertices $v\in P$, so $\rho_F\in \sigma_v(1)$ if and only if $v\in F$.

If $v\in F$ then $\ps{v,u_F}=-a_F$ by definition. $m_{\sigma_v}=v$ works.
\end{proof}

In terms of support functions
\begin{proposition}[]
If $P$ is a full dimensional lattice polytope then\footnote{that minimum exists because $P$ is compact and $\ps{\cdot,u}$ is continuous}
\[\vp_P:\funcDef{N_\R=\abs{\Sigma_P}}{\R}{u}{\min(\ps{m,u}\mid m\in P)}\]
is an integral support function on $\Sigma_P$. The corresponding divisor is $D_P$.
\end{proposition}
\begin{proof}
Recall that $\vp_{D_P}(u_F)=-a_F$. We have to check that $\vp\in SF(\Sigma_P,N)$ and that $\vp_P(u_F)=-a_F$.

If $v\in P$ is a vertex then $\sigma_v=\Cone(u_F\mid v\in F)$. If $u\in \sum_{v\in F}\la_F u_F\in \sigma_v$ then
\[\ps{m,u}=\sum_{v\in F}\la_F\ps{m,u_F}\geq -\sum_{v\in F}\la_F a_F\]
so $\vp(u)\geq -\sum_{v\in F}\la_F a_F$ and equality holds for $m=v$ because $v\in F$ and so $\ps{v,u_F}=-a_F$, so that value is the minimum, showing that $\vp(u)=-\sum \la_F a_F=\ps{v,u}$. This shows that $\vp\res{\sigma_v}$ is linear and that $\vp(u_F)=-a_F$.
\end{proof}


We can actually go the other way around now: for all $D\in \Div_{T_N(X_\Sigma)}$ we can define a polyhedron $P_D\subseteq M_\R$ (which might be unbounded and/or have non-integral vertices). This construction satisfies $P_{D_P}=P$ and gives a bijection
\[\cpa{\text{full dimensional lattice polytopes}}\leftrightarrow\cpa{(X_\Sigma,D)\mid \emat{\text{$X_\Sigma$ projective,}\\\text{$D$ ample and torus invariant}}}\]

Starting from $D=\sum a_\rho D_\rho$ we define
\[P_D=\cpa{m\in M_\R\mid \ps{m,u_\rho}\geq -a_\rho\ \forall \rho\in\Sigma(1)}.\]
This is the intersection of finitely many half-spaces a.k.a. a polyhedron.


The divisor $D$ corresponds to a sheaf $\Oc_X(D)$ and
\[\Oc_{X_\Sigma}(D)(X_\Sigma)=\cpa{f\in k{X_\Sigma}^\ast\mid \div f+D\geq 0}\cup\cpa 0=\bigoplus_{\div(t^m)+D\geq 0} k\cdot t^m\]
and $\div(t^m)=\sum \ps{m,u_\rho}D_\rho$, so $\div(t^m)+D\geq 0$ means that $\ps{m,u_\rho}+a_\rho\geq 0$. This shows that
\[\Oc_{X_\Sigma}(D)(X_\Sigma)=\bigoplus_{m\in P_D\cap M}k\cdot t^m\]
and indeed these are the characters that we used when defining $X_P$ for $P$ very ample. This corresponds to the general construction of projective embeddings coming from very ample line bundles.







