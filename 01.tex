\chapter{Algebraic tori and their actions}

\section{Basic definitions}
\begin{definition}[Algebraic group]
An \textbf{algebraic group} $G$ is a $k$-variety equipped with the the structure of a ``group object" in the category of $k$-varieties, i.e. we have two morphisms and a \textit{closed} point
\[m:G\times G\to G,\quad i:G\to G,\quad e\in G\]
that satisfy the usual group axioms ``diagrammatically".
\end{definition}

\begin{example}
Associativity can be expressed ``diagrammatically" as
% https://q.uiver.app/#q=WzAsNCxbMCwwLCJHXFx0aW1lcyBHXFx0aW1lcyBHIl0sWzEsMCwiR1xcdGltZXMgRyJdLFswLDEsIkdcXHRpbWVzIEciXSxbMSwxLCJHIl0sWzAsMiwiKG0saWRfRykiLDJdLFswLDEsIihpZF9HLG0pIl0sWzEsMywibSJdLFsyLDMsIm0iLDJdXQ==
\[\begin{tikzcd}
	{G\times G\times G} & {G\times G} \\
	{G\times G} & G
	\arrow["{(id_G,m)}", from=1-1, to=1-2]
	\arrow["{(m,id_G)}"', from=1-1, to=2-1]
	\arrow["m", from=1-2, to=2-2]
	\arrow["m"', from=2-1, to=2-2]
\end{tikzcd}\]
\end{example}

\begin{remark}
If $G=\Spec A$ is an affine variety, a structure of algebraic group is equivalent to a structure of \textbf{Hopf algebra} on $A$:
\begin{align*}
m:G\times G\to G\quad\longleftrightarrow& \quad\Delta:A\to A\otimes_k A\\
i:G\to G\quad\longleftrightarrow& \quad S:A\to  A\\
e:\Spec k\to G\quad\longleftrightarrow& \quad\e:A\to k
\end{align*}
and the homomorphisms $\Delta,\ S,\ \e$ satisfy the diagrammatic group axioms with the arrows reversed.
\end{remark}

\begin{remark}
If $G$ and $H$ are algebraic groups, $G\times H$ is also naturally an algebraic group. For example 
\[m_{G\times H}:\funcDef{(G\times H)\times (G\times H)}{G\times H}{((g_1,h_1),(g_2,h_2))}{(m_G(g_1,g_2),m_H(h_1,h_2))}.\]
\end{remark}

\begin{definition}[Homomorphism between Algebraic groups]
If $G,H$ are algebraic groups over $k$ then a homomorphism $f:G\to H$ is a morphism of $k$-varieties such that
% https://q.uiver.app/#q=WzAsNCxbMCwwLCJHXFx0aW1lcyBHIl0sWzEsMCwiSFxcdGltZXMgSCJdLFswLDEsIkciXSxbMSwxLCJIIl0sWzAsMiwibSIsMl0sWzEsMywibSIsMl0sWzIsMywiZiIsMl0sWzAsMSwiKGYsZikiXV0=
\[\begin{tikzcd}
	{G\times G} & {H\times H} \\
	G & H
	\arrow["{(f,f)}", from=1-1, to=1-2]
	\arrow["m"', from=1-1, to=2-1]
	\arrow["m"', from=1-2, to=2-2]
	\arrow["f"', from=2-1, to=2-2]
\end{tikzcd}\]
\end{definition}

\begin{remark}
If $G$ and $H$ are affine, the axioms of homomorphism dualize to what a homomorphism of Hopf algebras should be.
\end{remark}

\begin{remark}
All algebraic subgroups of an algebraic group are closed subvarieties.
\end{remark}

\bigskip

\begin{center}
	The first example of algebraic group we present is the multiplicative group
\end{center}


\begin{definition}[Multiplicative group]
The \textbf{multiplicative group}, denoted $\G_m$, is the $k$-variety $\A^1\bs \cpa0$ equipped with the morphisms
\[m:\funcDef{\G_m\times \G_m}{\G_m}{(a,b)}{ab}\]
\[i:\funcDef{\G_m}{\G_m}{a}{1/a}\]
\[e=1\in \A^1\nz\]
(we are identifying $\G_m=k^\ast$).
\end{definition}

\begin{remark}
$\G_m$ is affine, indeed $\A^1=\Spec k[x]$ and $\A^1\nz=\A^1\bs V(x)=D(x)$, thus 
\[D(x)=\Spec (k[x])_{x}=\Spec(k[x,x\ii])=\Spec k[x^{\pm1}].\]
If you are uncomfortable with ``$x\ii$" appearing you may simply think of this coordinate ring as
\[\frac{k[x,y]}{(xy-1)}.\]
\end{remark}




\begin{remark}
The multiplication $m:\G_m\times \G_m\to \G_m$ can be described as the morphism corresponding to the $k$-algebra homomorphism
\[\funcDef{k[x^{\pm1}]}{k[y^{\pm1}]\otimes_k k[z^{\pm1}]}{x}{y\otimes z}.\]
Similarly, the inverse corresponds to
\[\funcDef{k[x^{\pm1}]}{k[y^{\pm1}]}{x}{y\ii}\]
and the neutral element corresponds to\footnote{recall that a $k$-point $e$ of the variety $G$ can be seen as a morphism $\Spec k\to G$ with set-theoretic image $e$.}
\[\funcDef{k[x^{\pm1}]}{k}{x}{1}\]
\end{remark}





\begin{definition}[Algebraic tori]
The \textbf{standard $n$-dimensional algebraic torus over $k$} is $\G_m^n$. An \textbf{algebraic torus}\footnote{we may simply say ``torus" if no confusion can occur.} is an algebraic group $T$ which is isomorphic to $\G_m^n$ for some $n$.
\end{definition}

\begin{remark}
If $k=\C$ then $\G_m^n=(\C^\ast)^n$, which is homotopy equivalent to $(S^1)^n$. This $(S^1)^n$ is the ``maximal compact subgroup" and is the reason why these groups are called tori in the first place.
\end{remark}


\section{Cartier duality}
In this section we will define an equivalence of categories between finitely generated abelian groups\footnote{with no $p$-torsion if $p=\cha k\neq 0$} and a specific type of algebraic groups.
Under this correspondence, tori will be ``dual" to finitely generated free abelian groups.

\subsection{Group algebra and Cartier dual}
The first step is transforming general (finitely generated) abelian groups into (finite type reduced) algebras over $k$, the way we do this is via the

\begin{definition}[Associated group algebra]
If $M$ is a finitely generated abelian group, the \textbf{$k$-group algebra of $M$}, denoted by $k[M]$, is the $k$-vector space spanned formally by the basis $\cpa{t^m\mid m\in M}$ together with the multiplication induced by $t^mt^{m'}=t^{m+m'}$.
\end{definition}

\begin{example}
If $M=\Z^n$ then
\[k[\Z^n]=k[t^{(1,0,\cdots,0)}, t^{(-1,0,\cdots,0)},\cdots,t^{(0,\cdots,0,-1)}]=k[x_1^{\pm1},\cdots,x_n^{\pm1}],\]
which is the coordinate ring of $(\G_m)^n$.
\end{example}

\begin{fact}
These formulas give $k[M]$ a Hopf algebra structure for all finitely generated abelian groups $M$
\begin{align*}
\Delta:&\funcDef{k[M]}{k[M]\otimes_k[M]}{t^m}{t^m\otimes t^m}\\
S:&\funcDef{k[M]}{k[M]}{t^m}{t^{-m}}\\
\e:&\funcDef{k[M]}{k}{t^m}{1}
\end{align*}
\end{fact}
\begin{remark}
If we see $\G_m^n$ as $\Spec k[\Z^n]$ then the usual algebraic group structure is the one induced by the maps we just mentioned.
\end{remark}

\begin{remark}
If $M$ is finitely generated then $k[M]$ is of finite type over $k$. It turns out that it is also reduced when $M$ has no elements of order divided by the characteristic of $k$. 
\end{remark}


\begin{definition}[Cartier dual]
If $M$ is a finitely generated abelian group, $D(M):=\Spec k[M]$ is the \textbf{cartier dual} of $M$.
\end{definition}

Let us compute the cartier dual of another type of finitely generated abelian group:
\begin{example}
If $M=\znz n$ then the group algebra is
\[k[\znz n]=\frac{k[t]}{(t^n-1)}.\]
$\Spec k[\znz n]$ then is the closed subvariety (and subgroup) of $\G_m$ described by the equation $t^n=1$, i.e. the group of the $n$-th roots of unity $\mu_n$
\end{example}

\begin{definition}[Group of $n$-th roots of unity]
$\mu_n=D(\znz n)$.
\end{definition}

\begin{remark}
If $n=p=\cha k$ then $(t^p-1)=(t-1)^p$, so $\mu_p$ would be a point. To get any interesting geometric information in this case you need to allow nilpotens, stumbling into the teorritory of group schemes.
\end{remark}

Since we know the structure theorem for finitely generated abelian groups, let us consider the following

\begin{exercise}
$D(M\oplus N)=D(M)\times D(N)$.
\end{exercise}
\begin{proof}[Solution (Sketch).]
It is enough to note that $k[M\oplus N]=k[M]\otimes k[N]$ and this follows from the fact that
\[t^{(m,n)}=t^{(m,0)}t^{(0,n)}.\]
\end{proof}

It follows that 

\begin{proposition}\label{PrCartierDualOfGeneralFinitelyGeneratedAbelianGroup}
For a general finitely generated abelian group
\[M=\Z^n\oplus \znz{n_1}\oplus\cdots\oplus \znz{n_k}\]
the Cartier dual is
\[D(M)\cong \G_m^n\times\mu_{n_1}\times\cdots\times\mu_{n_k}.\]
\end{proposition}


Since we hope to find an equivalence of categories, let us try to understand another way in which we can view these types of algebraic groups.



\begin{remark}
$\GL_n$ is an algebraic group: It is a variety when seen as\footnote{the determinant is a homogeneous polynomial of degree $n$} $\A^{n^2}\bs V(\det)$ and it can be checked that matrix multiplication and inversion are morphisms of $k$-varieties.
\end{remark}

\begin{definition}[Diagonizable group]
An algebraic group is called \textbf{diagonalizable} if it is isomorphic to a (closed) subgroup of $\Diag_n\subseteq \GL_n$ for some $n$
\end{definition}

\begin{remark}
$\Diag_n\cong\G_m^n$ and the isomorphism is given by ignoring the entries which aren't on the diagonal.
\end{remark}

\begin{remark}
$D(M)$ is diagonalizable, because
\[D(M)\cong \G_m^n\times\mu_{n_1}\times\cdots\times\mu_{n_k}\subseteq \G_m^{n+k}\cong \Diag_{n+k}.\]
\end{remark}

\begin{proposition}
If $\vp:M\to N$ is a group homomorphism 
\[k[\vp]:\funcDef{k[M]}{k[N]}{t^m}{t^{\vp(m)}}\] 
is a $k$-algebra homomorphism and so $D(\vp)=\Spec(k[\vp]):D(N)\to D(M)$ is a morphism of $k$-varieties.

This is actually a homomorphism of algebraic groups and the association is functorial.
\end{proposition}

Cartier duality is that statement that
\[D:(\text{fin.gen.AbGps}_{\text{no $p$-tors}})\op\to (\text{Diag.AlgGps}),\]
where $p=\cha k$, is an equivalence of categories. To prove this fact we will build an inverse functor

\subsection{Character group}
To find the ``inverse" functor, we want to build a finitely generated abelian group from an algebraic group. The construction that will end up being what we want is the \textit{group of characters}


\begin{definition}[Character]
A \textbf{character} of an algebraic group $G$ is a homomorphism $\chi:G\to \G_m$. We denote the set of all characters $X(G)$.
\end{definition}
\begin{remark}
The characters of an algebraic group $G$ form an abelian group via:
\[\chi_1:G\to \G_m,\quad \chi_2:G\to \G_m\quad\leadsto\quad \chi_1\cdot\chi_2:G\xrightarrow{(\chi_1,\chi_2)}\G_m\times\G_m\xrightarrow{m}\G_m.\]
From now on $X(G)$ will always also have the group structure.
\end{remark}

\begin{example}
If $G=\G_m$ then for $k\in \Z$
\[\funcDef{\G_m}{\G_m}{a}{a^k}\]
is a character, which corresponds to
\[\funcDef{k[x^{\pm1}]}{k[x^{\pm1}]}{x}{x^k}\]
\end{example}

\begin{example}
If $G=\G_m^n$ and $(k_1,\cdots, k_n)\in\Z^n$ then
\[\funcDef{\G_m^n}{\G_m}{(a_1,\cdots,a_n)}{a_1^{k_1}\cdots a_n^{k_n}}.\]
We will see that these are all the characters on the torus.
\end{example}

\begin{example}
If $G=\GL_n$ the determinant is a character
\[\funcDef{\GL_n}{\G_m}{M}{\det M}\]
\end{example}


\begin{definition}[Group-like elements]
A \textbf{group-like element} in a Hopf algebra $A$ is an $a\in A$ such that $a$ is invertible and $\Delta(a)=a\otimes a$.
\end{definition}

\begin{lemma}\label{LmCharactersAreGroupLikeElements}
If $G=\Spec A$ is an affine algebraic group, characters of $G$ correspond to {group-like} elements of $A$.
\end{lemma}
\begin{proof}
A character $\chi:\Spec A\to \G_m$ corresponds to a homomorphism of Hopf algebras $k[x^{\pm1}]\to A$ which sends $x$ to some $a\in A$. The homomorphism is uniquely determined by $a$ so we just need to check which elements of $A$ can be the image of $x$. Since $x$ has an inverse, $a\in A^\ast$ and $\Delta(a)=a\otimes a$ because $\Delta(x)=x\otimes x$. On the other hand, an element which satisfies those properties does yield a Hopf-algebra homomorphism, so we are done.
\end{proof}

\subsection{Proof of Cartier duality}

\begin{remark}
Constructing the character group extends to a functor
\[X:(\text{AlgGps})\to (\text{AbGps})\]
via pullback, i.e. the map $f:G\to H$ becomes
\[X(f)\funcDef{X(H)}{X(G)}{\chi}{\chi\circ f}\]
\end{remark}


Now that we have built our candidate for the inverse functor, all we need to show that that the two compositions are naturally isomorphic to the identity.


\begin{proposition}
The map $M\to X(D(M))$ which to an element $m\in M$ assigns the character which corresponds to the Hopf-Algebra homomorphism
\[\funcDef{k[x^{\pm1}]}{k[M]}{x}{t^m}\]
is a natural isomorphism.
\end{proposition}

\begin{proof}
It is easy to check that $M\to X(D(M))$ is a group homomorphism.
\setlength{\leftmargini}{0cm}
\begin{itemize}
\item[$\boxed{\text{inj.}}$] If $m_1\neq m_2$ then $t^{m_1}\neq t^{m_2}$ and so the induced Hopf algebra homomorphisms are different. 
\item[$\boxed{\text{onto}}$] Given lemma (\ref{LmCharactersAreGroupLikeElements}), we just need to show that the only group-like elements of $k[M]$ are the $t^m$ for $m\in M$. Let us take any element $a=\sum_{m\in M}a_mt^m$ of $k[M]$ and impose that $\Delta(a)=a\otimes a$, then
\[\Delta(a)=\Delta\pa{\sum_{m\in M}a_mt^m}=\sum_{m\in M}a_m\Delta(t^m)=\sum_{m\in M}a_m t^m\otimes t^m\]
\[a\otimes a=\pa{\sum_{m\in M}a_mt^m}\otimes \pa{\sum_{m'\in M}a_{m'}t^{m'}}=\pa{\sum_{m,m'\in M}a_ma_{m'}t^m\otimes t^{m'}}.\]
Since the $t^m\otimes t^{m'}$ form a basis of $k[M]\otimes k[M]$, if $m\neq m'$ then $a_ma_{m'}=0$. Thus there exists at most one nonzero coefficient $a_{m_0}$ and $a=a_{m_0}t^{m_0}$, but $a$ must be invertible so $a_{m_0}\neq 0$. Also, again imposing the comultiplication condition, $a_{m_0}^2=a_{m_0}$, which implies that $a_{m_0}=1$ since it isn't $0$.
\end{itemize}
\setlength{\leftmargini}{0.5cm}
\end{proof}

\begin{corollary}
For $M=\Z^n$ we get $X(\G_m^n)\cong \Z^n$ and the characters are the ones we wrote above\footnote{$(a_1,\cdots, a_n)\mapsto a_1^{k_1}\cdots a_n^{k_n}$}.
\end{corollary}


Let us now consider the other composition:

\begin{remark}
There is a canonical map $\Spec A=G\to D(X(G))$.
\end{remark}
\begin{proof}
Let $\chi:G\to \G_m$ be a character of $G$. Upon composition with the inclusion $\G_m\subseteq \A^1$ we get a morphism in $\Hom(G,\A^1)$ and this set is canonically identified with $A$, so we get a map
\[\vp:X(G)\to A.\] 
This is a group homomorphism, which induces the desired map
\[\funcDef{k[X(G)]}{A}{t^m}{\vp(m)}.\]
\end{proof}

\begin{lemma}
Let $G$ be an abstract group (no algebraic structure) and $\K$ be any field, if we take $\phi_i:G\to \K^\ast$ distinct group homomorphisms then the $\phi_i$ are linearly independent in\footnote{not homomorphisms of any kind, just set theoretic functions. It is a $\K$-vector space by looking at the strucutre on the codomain.} $\Fun(G,\K)$
\end{lemma}
\begin{proof}
Let us assume by contradiction that we have a non-triavial relation $\sum a_i\phi_i=0$ for some $a_i\in \K$ and let's assume that this relation has minimal length.

By definition, $\sum a_i\phi_i(gh)=\sum a_i\phi_i(g)\phi_i(h)=0$ for all $g,h\in G$. Pick $\ol g\in G$ such that $\phi_1(\ol g)\neq \phi_2(\ol g)$ (which we can do because $\phi_1\neq \phi_2$). Setting $g=\ol g$ in the expression we get
\[\sum a_i\phi_i(\ol gh)=\sum \under{\in \K}{a_i\phi_i(\ol g)}\phi_i(h)=0\]
that is, $\sum {a_i\phi_i(\ol g)}\phi_i=0$ is an equality in $\Fun(G,\K)$. Multiplying the initial relation by $\phi_1(\ol g)$ we get
\[\sum a_i\phi_1(\ol g)\phi_i=0\]
subtracting the two functions we get
\[\sum a_i(\phi_1(\ol g)-\phi_i(\ol g))\phi_i=0\]
which is a shorter (look at $i=1$) non-trivial (look at $i=2$) reation, which is a contradiction.
\end{proof}

\begin{proposition}
If $G$ is diagonalizable then the homomorphism $G\to D(X(G))$ is an isomorphism and $X(G)$ is finitely generated. Moreover, if $\cha k=p\neq 0$ then $X(G)$ has no $p$-torsion.
\end{proposition}
\begin{proof}
Take a diagonalizable group $G$ and consider it as a closed subgroup $G\subseteq \G_m^n=\Diag_n$. Since it is \textit{closed} and $\G_m^n$ is affine, $G=\Spec A$ is also affine and we get a surjection\footnote{the surjection corresponds to taking $k[\Z^n]\to k[\Z^n]/I$ where $I$ is the ideal which defines $G$ as $V(I)\subseteq \G_m^n$.} $k[\Z^n]\to A$.

Now note that we have $\Z^n\cong X(\G_m^n)\to X(G)$ and the surjection above factors 
\[k[\Z^n]\to k[X(G)]\to A\]
since the composition is surjective, $k[X(G)]\to A$ is also surjective. To conclude the first part of the proof then, we just need to show that the map is also injective, but this follows from the lemma.

\medskip

Now we concern ourselves with finite generation. Because of the isomorphism we just proved, the factorization
\[k[\Z^n]\to k[X(G)]\to A\]
now shows that $k[\Z^n]\to k[X(G)]$ is surjective because $k[\Z^n]\to A$ was. This lets us conclude that $\Z^n\to X(G)$ is surjective (and thus $X(G)$ is finitely generated) because otherwise $k[\Z^n]\to k[X(G)]$ wouldn't be.


Suppose now that $0\neq p=\cha k$. Let $\chi\in X(G)$ be a $p$-torsion character, i.e. $\chi^p=1$, that is, $\chi(g)^p=1$ for all $g\in G$. Because $x^p-1=(x-1)^p$ in characteristic $p$, $\chi(g)=1$ for all $g\in G$, showing that $\chi=1$ and thus the absence of $p$-torsion.
\end{proof}




\begin{corollary}
A connected subgroup of a torus is a torus.
\end{corollary}
\begin{proof}
If $G\subseteq \G_m^n$, from the proposition we get that 
\[G=D(X(G))\cong \G_m^k\times \mu_{n_1}\times\cdots\times \mu_{n_r},\] 
but if $G$ is connected then all $n_i$ must be $1$ because otherwise that product would be disconnected.
\end{proof}

Having now verified both compositions we may formally state Cartier duality as a theorem now
\begin{theorem}[Cartier duality]\label{ThCartierDuality}
The functor
\[D:(\text{fin.gen.AbGps}_{\text{no $p$-tors}})\op\to (\text{Diag.AlgGps}),\]
where $p=\cha k$, is an equivalence of categories. The inverse functor is $X$.
\end{theorem}


\begin{remark}
If we allow group schemes the problem with $p$-torsion doesn't come up.
\end{remark}















\begin{proposition}
Let $f:T_1\to T_2$ be a homomorphism of tori, then the image is also a torus.
\end{proposition}
\begin{proof}
Since $T_1\to D(X(T_1))$ and $T_2\to D(X(T_2))$ are isomorphisms and the appropriate diagrams commute, we have that $f$ is induced by the corresponding homomorphism $M_2\to M_1$ where $M_1=X(T_1)$ and $M_2=X(T_2)$.

Let $K=\ker(M_2\to M_1)$ and note that $M_2\onto M_2/K\inj M_1$. We claim that $L:=\ker(k[M_2]\to k[M_1])$ is the ideal $I=(t^m-t^{m'}\mid \vp(m)=\vp(m'))$:
\setlength{\leftmargini}{0cm}
\begin{itemize}
\item[$\boxed{I\subseteq L}$] It suffices to note that the generators of $I$ lie in $L$, indeed $t^m-t^{m'}\mapsto t^{\vp(m)}-t^{\vp(m')}=0$.
\item[$\boxed{L\subseteq I}$] Let $\sum_{m\in M_2}a_mt^m$ be a general element of $L$, then
\[\sum_{n\in M_1}\pa{\sum_{m\in \vp\ii(n)}a_m}t^n=0\overset{\text{lin.ind.}}\Longrightarrow \sum_{m\in \vp\ii(n)}a_m=0\quad \forall n\in M_1\]
For a fixed $n$, if $a_{m_1},a_{m_2}\neq 0$ for some $m_1,m_2\in \vp\ii(n)$ (if all are $0$ ok, just one nonzero is impossible given that the whole sum is zero) we can write
\[\sum a_m t^m=\under{\in I}{a_{m_1}(t^{m_1}-t^{m_2})}+\under{\text{removed term with }t^{m_1}}{(a_{m_2}+a_{m_1})t^{m_2}+\sum_{m\neq m_1,m_2}a_mt^m}\]
iterating this process shows the other inclusion.
\end{itemize}
\setlength{\leftmargini}{0.5cm}
Thus we can factor $k[M_2]\to k[M_1]$ as $k[M_2]\onto k[M_2]/I\inj k[M_1]$. One can check that\footnote{taking the image is a colimit, $M\mapsto k[M]$ is a left adjoint functor, indeed $\Hom_{AbGps}(M,A)\cong \Hom_{k-Alg}(k[M],A)$} $k[M_2]/I=k[M_2/K]$. Since $M_2/K\inj M_1$ and $M_1$ is a free abelian group, $M_2/K$ is also free and thus
\[T_1\onto \under{\text{torus}}{\Spec k[M_2/K]}\inj T_2\]
where to check injectivity we use $k[M_2]\to k[M_2/K]$ surjective and to check surjectivity, because subgroups are closed, it is enough to check for dominance and indeed $k[M_2/K]\to k[M_1]$ is injective.
\end{proof}

\begin{remark}
We could have just said that the image is a connected subgroup of a torus and thus is also a torus, but the proof given is more instructive.
\end{remark}











